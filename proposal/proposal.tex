\documentclass[11pt]{article}
\bibliographystyle{siam}

\title{Mixed-Gambles Task}
\author{

  Chang, Siyao\\
  \texttt{changsiyao}
  \and
  Hsieh, Benjamin\\
  \texttt{BenjaminHsieh}
  \and
  Gong, Boying\\
  \texttt{boyinggong}
  \and
  Qiu, Brian\\
  \texttt{brianqiu}
  \and
  Zhu, Jiang\\
  \texttt{pigriver123}
}

\begin{document}
\maketitle

Our paper name: The Neural Basis of Loss Aversion in Decision-Making Under Risk
\cite{Tom2007LossAversion}.

The mixed gambles paper examines neural and behavioral loss aversion, the
sensitivity to losses and gains. 16 subjects are given a total of 256
different loss/gain trials in 3 runs and their behavior and fMRI scans are
recorded. 

The paper finds that people are roughly twice as sensitive to losses as they
are to gains. The prior belief is that negative emotions such as fear is the
main contributor to people being more sensitive to losses and that the
increasing potential losses would increase brain activities in the parts that
controls negative emotions. However, the paper discovers that potential loss
and gain are coded by the same process in the brain, meaning that potential
gain corresponds to a higher activation of some part of the brain and
potential loss corresponds to a lower activation of that same part. The paper
also linked behavioral loss aversion to neural and suggests that individual
differences in behavioral are due to individual differences in neural
sensitivity. The paper goes on to specify which regions of the brain are or
are not activated (the amygdala was largely inactivated to their surprise).

We have downloaded the data and confirmed the number of subjects (16) as well
as the task condition files. For each subject, we have anatomy, behavorial,
BOLD, and model files; with BOLD data having 4 dimensions. We have a preliminary 
understanding of the data. 

Our objective is to replicate the study as well as we can. We will go through
each subject and measure each run's BOLD signals at different voxels. We will
use this data to create a gain/loss matrices to show the acceptance
probability and response time to all the combinations of gain and losses. We
will also try to reproduce the parametric approach of the whole-brain
analyses, look at the effect of potential gains and losses to brain
activation, and create visualizations to show the brain parts with significant
change in fMRI signals. With the regression result, we would be able to
calculate the neural loss aversion, and compare it to the behavior loss
aversion (calculated from a separate logistic model). We hope that through
these procedure, we can understand the topic of the paper better, as well as
checking/verifying the conclusion made by the authors. 

\bibliography{proposal}

\end{document}
