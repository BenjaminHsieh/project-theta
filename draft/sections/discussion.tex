\subsection{Further Research}

We fit a linear regression model combining behavioral and BOLD data to examine 
the relationship of correlation between neural activity and behavioral 
response, we use another method which is different from what is mentioned in 
the paper. We add the behavioral response to the regression model on BOLD data 
as a predictor. We use the original 4-level response as stated below. \\ 

Moreover, if the three tests we do for the linear regression model is bad. We 
can plot the independents and the dependents on plots to see whether they fit a 
model that is different from linear regression models. There may be another 
reason why the performance of linear regression models are bad which is that we 
simplify our model that we didn’t try a mixed model as the researchers in the 
paper did.

\begin{tabular}{lllll}
\hline
behavioral response & strongly accept & weakly accept & weakly reject & 
strongly reject\\ 
\hline
$X_{behav}$ & 1 & 2 & 3 & 4 \\
\hline
\end{tabular}

And the models are following:

\begin{equation}
Y_{i} = \beta_{i, 0} + \beta_{i, behav} * X_{behav} + \epsilon_i
\end{equation}

However, since the response and level of loss and gain are potentially 
correlated, we might need to use stepwise regression to choose the best 
predictor from the regression model presented above.

\subsubsection{Brain Map}
We decided to not use the filtered fMRI data due to its large data size (almost
15 GB); the raw was computational less exhaustive and more managable. Yet one of
the drawbacks of not using the filtered data is that we lack the brain 
registration to a standard anatomical template (the MNI template). Thus, 
much of our analysis was done by comparing subjects separately and using 
visuals to help motivate analysis. In the future, it may be of interest to map
beta estimates from each subject onto the standard brain template to produce a 
a unified look at the neural loss aversion. 

\subsubsection{Other Potential Analysis}
\par Looking at our data of subjects, it may be of interest to consider a 
demographic grouping by gender because our dataset contains the demographics of 
our 16 subjects, with the extra information of gender and age. A question to 
potentially address: Is there a significant difference to loss aversion across
genders?

\par Additionally, it is interesting to see that the behavorial data contains a 
column for the response time of each gambling task. To further explore how 
decision are made in gambling task, we can use the response time as one of the 
logistic regressors. Through step-wise or criterion based model selection 
methods (eg. AIC and backward elimination), we can attempt to find the best 
regressors that influence loss aversion to most.
