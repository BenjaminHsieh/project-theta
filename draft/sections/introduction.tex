\section{Introduction}
The experiment conducted in the paper itself involved giving 16 subjects a total of 256 combinations of gain/loss in dollars with a 50 percent chance of winning. The subject's decision of whether to accept or reject each proposed gamble was recorded as well as the their brain activity in the fMRI machine as they made their decision. The neural activity is measured in BOLD signals.

\par 
We determine the beta coefficients for each voxel in the subject's brain based on a linear regression on the BOLD signal data. We use these coefficients as evidence for increasing activation in certain voxels in the brain when the subject is making a decision to accept or reject, and so in this way we have a proxy for which regions of the brain are involved with loss-aversion decision-making. We also determine the p-values to validate these results.

\par 
We also run a logistic regression to determine each subject's willingness to accept or reject based on the values of potential loss and gain. This is the measure of behavioral loss aversion, and in the end we correlate this with ou neural loss aversion, making it clear that increasing gains correspond to increasing activity in certain parts of the brain.  

\par 
We also aim to investigate briefly if there are any statistically significant variations between the male and female subjects in addition to showing the brain activations through various graphs.

