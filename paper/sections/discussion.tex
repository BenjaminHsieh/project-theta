\section{Discussion}

\subsection{Issues with analyses and potential solutions}

\subsubsection{Selecting specific regions to further explore 
correlation between neural and behavioral activity}

\indent \indent Since we have no knowledge on the sections of brain that might 
experience large difference in activation, it is hard for us to identify the 
specific regions in terms of anatomy to explore the correspondence between neural 
and behavioral loss aversion.

There are two potential ways to deal with this issue. The first one is a 
top-down appraoch to isolate potential regions of analysis. This method 
requires us to read more paper and related articles to better our anatomical 
understanding of the specifc brain regions that are more likely to react in our 
given scenario -- loss aversion faced with potential gain and loss combinations.

Another way is a bottom-up approach. In this method, we can fit a regression for 
every part of the brain and look for the areas with higher correspondence 
(higher slope). Then, we select and graph a few areas with the most significant 
positive or negative correlation between the parametric response to potential 
losses and behavioral loss aversion (ln(λ)) across participants by knowing the 
standard template of neural response. This second method is what we have attempted 
to complete in our paper, conducting whole brain analysis using standardized 
filtered data (provided by Matthew Brett) and producing significant results to 
compare with those of Tom et al. Please see \textit{Methods} and \textit{Results} 
for further discussion of analysis procedures and results. 

\subsubsection{Run Time Issues}

One of the main issues is the run time of our analysis. More specifically, 
the scripts for running the mixed effects model is very time-exhaustive. On
our laptop, that script alone took over a day (Boying can confirm). Perhaps
this is due to the lower processing speeds on our laptops compared with the
standard desktop/research computing hardware used in standard fMRI research
settings. Additionally, by using the ds005 dataset with the standard mni 
template, the regression scripts takes more than 4 hours as well. This is due
to the significantly larger size of the filtered data (almost 15 GB). 
Nonetheless, we managed to complete our analysis and generated the figures 
shown in this paper in a reproducible pipeline. We also note that hardware 
improvements is not the only way to address run time issues. In terms of 
software, code optimization may be another way to decrease the run time. 
Specifically, we can explore a better balance between performance and clarity
with more experience in scientific computing and practice with code 
optimization.

\subsection{Further Research}

\par Looking at our data of subjects, it may be of interest to consider a 
demographic grouping by gender because our dataset contains the demographics of 
our 16 subjects, with the extra information of gender and age. A question to 
potentially address: Is there a significant difference to loss aversion across
genders?

\par Additionally, it is interesting to see that the behavorial data contains a 
column for the response time of each gambling task. To further explore how 
decision are made in gambling task, we can use the response time as one of the 
logistic regressors. Through step-wise or criterion based model selection 
methods (eg. AIC and backward elimination), we can attempt to find the best 
regressors that influence loss aversion to most.
