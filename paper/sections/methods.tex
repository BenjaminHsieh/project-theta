\section{Methods}

\subsection{Models and analysis}

In this section, we present the models we used to find the relationship between behavioral and neural loss aversion cross participants as well as how participants react to different loss and gain levels. For behavioral data, we fit the logistic regression models for each subject and use the coefficients of loss and gain to calculate the behavioral loss aversion levels. For neural data, we fit both linear multiple regression models and mixed-effects models in order to collapse three runs for each subject into one model. We analysis the fMRI data use both models and compare the results we obtained. Neural loss aversion levels are calculated using the coefficients of loss and gain. 
%% Moreover, we may use the robust regression to reduce the influence of outliers.
%% At last, we find out find the correlation between the behavioral and neural loss aversion cross participants using the loss aversion data obtained from the aforementioned models.

\subsubsection{Behavioral analysis}

We fit a Logistic regression model on the behavioral data to examine how the response of individuals relates to the size of potential gain and loss of a gamble. Originally there are four acceptability judgments categories, here we collapse the categories into binary response(accept/reject). Following is the model:

\begin{equation}
\textrm{log}\frac{p(X)}{1-p(X)} = \beta_0 + \beta_{loss} *X_{loss} + \beta_{gain} * X_{gain}
\end{equation}

where $X_{loss}$ and $X_{gain}$ are the potential loss and gain value separately, $Y_{resp}$ is a categorical independent variable representing the subjects' decision on whether to accept or reject the gambles:

\begin{displaymath}
Y_{resp} = \left \{ \begin{array}{ll}
1 & \textrm{If the subject accepted the gamble.} \\
0 & \textrm{If the subject rejected the gamble.}
\end{array} \right .
\end{displaymath}

Then we calculate the the behavioral loss aversion ($ \lambda $) for each subject as follows, note that for simplicity, we collapse 3 runs into one model for each participant.

\begin{equation}
\lambda = -\beta_{loss} / \beta_{gain}
\end{equation}

We use $\lambda$ as the metric for the degree of behavioral loss aversion for each participant. 

\subsubsection{Linear Regression on fMRI data}

For each voxel $i$, we fit a multiple linear model:

\begin{equation}
Y_{i} = \beta_{i, 0} + \beta_{i,1} *X_{ldrift} + \beta_{i, 2} * X_{qdrift} + \beta_{i, loss} *X_{loss} + \beta_{i, gain} * X_{gain}  + 
\epsilon_i
\end{equation}

where $Y_{i}$ is the BOLD data of voxel $i$,  $X_{ldrift}$ and $X_{qdrift}$ are linear and quadratic drift terms. For each voxel, we calculate the 
neural loss aversion $\eta_i$:

\begin{equation}
\eta_i = (-\beta_{loss}) - \beta_{gain}
\end{equation}

Using the voxelwise neural loss aversion, we do a region-specific analysis on 
BOLD data for each participant. That is, we plot a heat map of $\eta_i$ and  
$\beta_{i, loss}$, $ \beta_{i, gain}$ for each participant to find out the 
regions with significant activation and regions which show a significant 
positive or negative correlation with increasing loss or gain levels.

\subsubsection{Mixed-effects model on fMRI data}

%% do anova first to see if we should do a mixed model or not.

The fact that we have 3 runs of data for each participants leads us to consider using mixed effects model to analysis the data set. For each voxel $i$, we fit the following mixed-effects models, note that here we only include the intercept term for random effects. 

\begin{equation}
Y_{i, k} = \beta_{i, 0} + \beta_{i,1} *X_{ldrift} + \beta_{i, 2} * X_{qdrift} +  \beta_{i, loss} *X_{loss} + \beta_{i, gain} * X_{gain}  + \gamma _{i, k} + \epsilon_{i, k}, \quad k =1, 2, 3
\end{equation}

Then we calculate the neural loss aversion level and plot heat maps same as the about section for multiple linear regressions and compare the results of two models.

\subsubsection{Whole brain analysis of correlation between 
neural activity and behavioral response across participants}

We then apply the above model on the standard brain to analysis the neural 
activity and behavioral response across participants. For each participant, 
we pick up several regions with highest activation level, calculate the mean 
neural loss aversion $\bar{\eta}$ within these specific region. Thus we could 
examine the relationship between neural activity and behavioral using the 
following regression model:

\begin{equation}
\lambda = \alpha_0 + \alpha_1 * \eta + \epsilon
\end{equation}

where the sample size is the number of participants(16).

\subsubsection{Cross-validation}

We fit linear models for each voxel for each participant. For each linear
model, we do a k-fold cross-validation. Since the sample size for each linear 
regression model range from 80-90, we choose to use 10 fold cross-validation,
which means the original sample is randomly pertitioned into 10 equal sized 
subsamples. \\
In the behavioral analysis using Logistic regression, since the responce 
variables are binary, we calculate the misclassification error rate to 
summarize the fit. In the neural linear regression model using BOLD data, we 
use the mean squared error to summarize the errors.

\subsubsection{Inferences on Data}

After fitting regression models on our BOLD and behavioral data, we would try 
assessing and validating our models. In order to do this, we would calculate 
for the residual sum of squares for our model. We have to do three tests for 
the model. The first one is that we calculate the t-statistics and p-value for 
our beta coefficients to check whether our beta parameters are statistically 
significant at a significance level of 5\%. The second one is that we calculate
the residuals of this linear model and check whether it follows a normal 
distribution. The third one is that we calculate the R-Squared value and the 
adjusted R-squared value to see whether the values are good for the linear 
regression model.\\


\subsection{Explanation on model simplification}

\subsubsection{Use of Data}
\indent \indent First of all, for simplicity reasons, we are not using all the 
regressors the paper used. The model in the paper performed regression on the 
BOLD data with gain, loss and euclidean distance to indifference. In our model,
we are leaving out the regressor euclidean distance to indifference. The paper 
and its supplement material didn't document the exact way the authors 
calculated this parameter; we are having a hard time reproducing this 
parameter. Therefore, we decide to leave out this parameter when doing our own 
regression.

\subsubsection{Perform multiple regression and mixed-effects model on BOLD data}

\indent \indent We fit two separate models on neural data. In the 
original data analysis, the authors performed a mixed effect model when 
regressing the potential gain and loss values against the BOLD data across 
runs, since there are three different runs for each subject and the authors 
were trying to incorporate all three runs into one model. The mixed effect 
model adds a random effects term, which is associated with individual 
experimental units drawn at random from a population. In this case, it 
measures the difference between the average brain activation in run i and the 
average brain activation in all three runs.

While fitting the mixed-effects model using package \emph{statmodels}, we 
are simplifying the model because it is much easier to perform a simple 
linear regression in python. For the multiple regression part, we write functions 
to calculate coefficients and do multiple inferences on the regression model.

After looking at the initial result from our linear regression model, we can 
decide whether we want to further explore the relationship between the 
dependent variable (BOLD data) and the independent variables (gain and loss).
