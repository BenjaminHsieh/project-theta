\section{Methods}

\subsection{Models and analysis}

We use linear models to find the relationship between behavioral and nueral 
loss aversion cross participants as well as how participants react to different 
loss and gain level. Below we illustrate our model using simple multiple linear 
regression form. We may implement a mixed-effects model treating partipants as
a random effect. Moreover, we may use the robust regression to reduce the 
influence of outliers.

\subsubsection{Behavioral analysis}

We fit a Logistic regression model on the behavioral data to examine how the 
response of individuals relates to the size of potential gain and loss of a 
gamble. Following is the model:

\begin{equation}
logit(Y_{resp}) = \beta_0 + \beta_{loss} *X_{loss} + \beta_{gain} * X_{gain}  + 
\epsilon
\end{equation}

where $X_{loss}$ and $X_{gain}$ are the potential loss and gain value 
seperately, $Y_{resp}$ is a categorical independent variable representing 
the subjects' decision on whether to accept or reject the gambles:

\begin{displaymath}
Y_{resp} = \left \{ \begin{array}{ll}
1 & \textrm{If the subject accepted the gamble.} \\
0 & \textrm{If the subject rejected the gamble.}
\end{array} \right .
\end{displaymath}

Then we calculate the the behavioral loss aversion ($ \lambda $) for each 
subject as follows, note that for simplicity, we collapse 3 runs into one model
for each participant.

\begin{equation}
\lambda = -\beta_{loss} / \beta_{gain}
\end{equation}

We use $\lambda$ as the metric for the degree of loss aversion for each 
participant. We have used R to fit the Logistic model, just as what the 
authors did in the paper, and we achieved almost the same results as the paper 
presented.

\subsubsection{Linear Regression on BOLD data}

For each voxel $i$, we fit a multiple linear model:

\begin{equation}
Y_{i} = \beta_{i, 0} + \beta_{i, loss} *X_{loss} + \beta_{i, gain} * X_{gain}  + 
\epsilon_i
\end{equation}

where $Y_{i}$ is the BOLD data of voxel $i$. For each voxel, we calculate the 
neural loss aversion $\eta_i$:

\begin{equation}
\eta_i = (-\beta_{loss}) - \beta_{gain}
\end{equation}

Using the voxelwise neural loss aversion, we do a region-specific analysis on 
BOLD data for each participant. That is, we plot a heat map of $\eta_i$ and  
$\beta_{i, loss}$, $ \beta_{i, gain}$ for each participant to find out the 
regions with significant activation and regions which show a significant 
positive or negative correlation with increasing loss or gain levels.

\subsubsection{Whole brain analysis of correlation between 
neural activity and behavioral response across participants}

We then apply the above model on the standard brain to analysis the neural 
activity and behavioral response across participants. For each participant, 
we pick up several regions with highest activation level, calculate the mean 
neural loss aversion $\bar{\eta}$ within these specific region. Thus we could 
examine the relationship between neural activity and behavioral using the 
following regression model:

\begin{equation}
\lambda = \alpha_0 + \alpha_1 * \eta + \epsilon
\end{equation}

where the sample size is the number of participants(16).

\subsubsection{Cross-validation}

We fit linear models for each voxel for each participant. For each linear
model, we do a k-fold cross-validation. Since the sample size for each linear 
regression model range from 80-90, we choose to use 10 fold cross-validation,
which means the original sample is randomly pertitioned into 10 equal sized 
subsamples. \\
In the behavioral analysis using Logistic regression, since the responce 
variables are binary, we calculate the misclassification error rate to 
summarize the fit. In the neural linear regression model using BOLD data, we 
use the mean squared error to summarize the errors.

\subsubsection{Inferences on Data}

After fitting regression models on our BOLD and behavioral data, we would try 
assessing and validating our models. In order to do this, we would calculate 
for the residual sum of squares for our model. We have to do three tests for 
the model. The first one is that we calculate the t-statistics and p-value for 
our beta coefficients to check whether our beta parameters are statistically 
significant at a significance level of 5\%. The second one is that we calculate
the residuals of this linear model and check whether it follows a normal 
distribution. The third one is that we calculate the R-Squared value and the 
adjusted R-squared value to see whether the values are good for the linear 
regression model.\\


\subsection{Explanation on model simplification}

\subsubsection{Use of Data}
\indent \indent First of all, for simplicity reasons, we are not using all the 
regressors the paper used. The model in the paper performed regression on the 
BOLD data with gain, loss and euclidean distance to indifference. In our model,
we are leaving out the regressor euclidean distance to indifference. The paper 
and its supplement material didn't document the exact way the authors 
calculated this parameter; we are having a hard time reproducing this 
parameter. Therefore, we decide to leave out this parameter when doing our own 
regression.

\subsubsection{Simplification of regression on BOLD data}

\indent \indent We plan on simplifying the model on neural data. In the 
original data analysis, the authors performed a mixed effect model when 
regressing the potential gain and loss values against the BOLD data across 
runs, since there are three different runs for each subject and the authors 
were trying to incorporate all three runs into one model. The mixed effect 
model adds a random effects term, which is associated with individual 
experimental units drawn at random from a population. In this case, it 
measures the difference between the average brain activation in run i and the 
average brain activation in all three runs.

We are simplifying the model because it is much easier to perform a simple 
linear regression in python. In addition, we do not have a great deal of 
understanding of fMRI data, so simple linear model would suffice when we are 
only performing exploratory data analysis and looking for obvious pattern in 
the data.

After looking at the initial result from our linear regression model, we can 
decide whether we want to further explore the relationship between the 
dependent variable (BOLD data) and the independent variables (gain and loss) 
and whether we want to continue to fit a mixed effect model.

\subsection{Issues with analyses and potential solutions}

\subsubsection{Selecting specific regions to further explore 
correlation between neural and behavioral activity}

\indent \indent Since we have no knowledge on the sections of brain that might 
experience large difference in activation, it is hard for us to pick the 
regions to deeper explore the correspondence between neural and behavioral loss 
aversion.

There are two potential ways to deal with this issue. The first one is to read 
more paper and related articles to learn which parts of the brain are likely to 
react in our given scenario -- faced with potential gain and loss combinations. 
Another way to deal with the issue to to fit a regression for every part of the 
brain and look for the areas with higher correspondence (higher slope). Then, 
we select and graph a few areas with the most significant positive or negative 
correlation between the parametric response to potential losses and behavioral 
loss aversion (ln(λ)) across participants.

\subsubsection{Producing heat map}

\indent \indent Another issue that we are facing during our project is finding 
the same region to plot for each participant. We see that each region of the 
brain has its own standard coordinates. However, without much knowledge of 
fMRI, we are not sure how to use these standard coordinates to locate the 
regions of the brain.

From our understanding, each subject's brain is mapped onto a standard brain 
and we then use the coordinates for the standard brain to extract data from 
the areas we are interested in. However, currently, we don't have the skill to 
perform this step.

\subsubsection{Further Research}

We fit a linear regression model combining behavioral and BOLD data to examine 
the relationship of correlation between neural activity and behavioral 
response, we use another method which is different from what is mentioned in 
the paper. We add the behavioral response to the regression model on BOLD data 
as a predictor. We use the original 4-level response as stated below. \\ 

Moreover, if the three tests we do for the linear regression model is bad. We 
can plot the independents and the dependents on plots to see whether they fit a 
model that is different from linear regression models. There may be another 
reason why the performance of linear regression models are bad which is that we 
simplify our model that we didn’t try a mixed model as the researchers in the 
paper did.

\begin{tabular}{lllll}
\hline
behavioral response & strongly accept & weakly accept & weakly reject & 
strongly reject\\ 
\hline
$X_{behav}$ & 1 & 2 & 3 & 4 \\
\hline
\end{tabular}

And the models are following:

\begin{equation}
Y_{i} = \beta_{i, 0} + \beta_{i, behav} * X_{behav} + \epsilon_i
\end{equation}

However, since the response and level of loss and gain are potentially 
correlated, we might need to use stepwise regression to choose the best 
predictor from the regression model presented above.

